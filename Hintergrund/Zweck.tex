\section*{Zweck des Dokuments}
Dieses Dokument gehört zu den Dokumenten der FH Aachen im Verbundprojekt ''Neue Elektronik- und Kommunikationssysteme für den intelligenten, vernetzten Güterwagen'' - Güterwagen 4.0 - Teilvorhaben: ''Entwicklung von Grundlagen für Aktorik, Sensorik und Predictive Maintenance''.\par
Dieses Dokument zeigt das Konzept des Güterwagens in genau diesen Punkten, also in den Aspekten 'elektrische und pneumatische Aktorik', 'Sensorik' und 'Daten und Kommunikation'. Diese Punkte bilden die Basis für die Arbeitspakete 3, 4 und 5. Besonders Wert wird auf das Verständnis des Konzeptes im Allgemeinen und Speziellen gelegt.
Dieses Dokument ist also als Teil folgenden Arbeitspakete zu sehen:
\begin{itemize}
    \item AP 3: Entwicklung Aktorik
    \begin{itemize}
        \item 3a: Konzeptentwicklung Aktorik und Aktor-Steuerung
    \end{itemize}
    \item AP 4: Entwicklung Sensorik
    \begin{itemize}
        \item 4a: Konzeptentwicklung Sensorik
    \end{itemize}
    \item AP 5: Datenkommunikation
    \begin{itemize}
        \item 5a: Entwicklung eines Hardware- und Software-Konzeptes für die Kommunikation
    \end{itemize}
\end{itemize}
