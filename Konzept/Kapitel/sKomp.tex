\section{Konzept sensorischer Komponenten}
In diesem Kapitel werden die sensorischen Komponenten des \gls{Gueterwagen 40} einzeln beschrieben. Dieses Kapitel ist zugehörig zum Arbeitspaket 4 - Entwicklung Sensorik; genauer AP 4a: Konzeptentwicklung Sensorik und behandelt den sensorischen Teil des \gls{Gueterwagen 40} inklusive Condition Monitoring.\par
Die sensorischen Komponenten sind in Abbildung \ref{fig:sKomp} in rot farblich aufgezeigt.
\begin{figure}[hbt]
    \centering
    \input{Bilder/SchemaSKomp.tex}
    \caption{Sensorische Komponenten des Gesamtsystems}
    \label{fig:sKomp}
\end{figure}
Für den \gls{Demonstrator} ist eine Teilausstattung mit Sensoren geplant. Diese dienen zum Auslesen von Aktoren sowie zur Zustandsüberwachung des Wagens.\par
Folgende Zustände von \gls{40-Komponenten} sollen überwacht werden:
\begin{itemize}
    \item Batteriestand
    \item Pneumatische Kupplung
    \item Steuerventilstellung
    \item Relaisventilstellung
    \item Stellung der \gls{HL}-Ventile
    \item Stellung der Feststellbremse
    \item C-Druck
\end{itemize}
Zusätzlich soll auch der Zustand des Wagens überwacht werden. Dafür sind folgende Sensoren vorgesehen:
\begin{itemize}
    \item Lagertemperatur
    \item Beschleunigungen in x-, y- und z-Richtung
    \item Laufleistung/Radumdrehung
    %\item Flachstellendetektion an jedem Rad
    %\item Lagertemperatur an jedem Lager
    %\item Geschwindigkeit an jeder Achse
    %\item Laufleistung an jeder Achse
    %\item Stoßsensor an jeder Wagenseite
    %\item Bremsbelagüberwachung an jeder Bremszange
\end{itemize}

\begin{comment}
\subsection{Konzept}
Condition Monitoring\\
Laderaumtemperatur\\
Türüberwachung\\
Stöße\\
Flachstellendetektion\\
Geschwindikeit\\
Laufleistung\\
Bremsbelag\\
Temperaturen\\
Batteriestand
\end{comment}
